\section{Reference Frames}

\subsection{Celestial To Terrestrial Transformation}
\label{sec:itrs-to-gcrs}

To transform a a state vector from \gls{gcrf} to \gls{itrf}, we can use: 
\begin{align}\label{eq:vallado2201}
  \vec{r}_{ITRF} &= \bm{W}^T \bm{R}^T \bm{Q}^T \vec{r}_{GCRF} \\
  \vec{v}_{ITRF} &= \bm{W}^T \left[ \bm{R}^T \bm{Q}^T \vec{v}_{GCRF} - \vec{\omega _{\Earth}} \times \vec{r}_{TIRS} \right]
\end{align}

while the inverse transformation can be achieved via:
\begin{align}\label{eq:vallado2202}
  \vec{r}_{GCRF} &= \bm{Q} \bm{R} \bm{W} \vec{r}_{ITRF} \\
  \vec{v}_{GCRF} &= \bm{Q} \bm{R} \left[ \bm{W} \vec{v}_{ITRF} + \vec{\omega _{\Earth}} \times \vec{r}_{TIRS} \right]
\end{align}

The same result (for the velocity components) can be obtained using:
\begin{align}
  \vec{v}_{ITRF} &= \bm{W}^T \dot{\bm{R}}^T \bm{Q}^T \vec{r}_{GCRF} + \bm{W}^T \bm{R}^T \bm{Q}^T \vec{v}_{GCRF} \\
  \vec{v}_{GCRF} &= \bm{Q} \dot{\bm{R}} \bm{W} \vec{r}_{ITRF} + \bm{Q} \bm{R} \bm{W} \vec{v}_{ITRF}
\end{align}

where 
\begin{equation}\label{eq:vallado2221}
\dot{\bm{R}} = \begin{pmatrix} -\omega_{\Earth} \sin \theta _{ERA} & -\omega_{\Earth} \cos \theta _{ERA} & 0 \\
  \omega_{\Earth} \cos \theta _{ERA} & -\omega_{\Earth} \sin \theta _{ERA} & 0 \\
  0 & 0 & 0 \end{pmatrix}
\end{equation}


\begin{figure}[ht]
\centering
  \begin{tikzpicture}[
      node distance=1.5cm and 2cm,
      every node/.style={font=\sffamily},
      rect/.style={rectangle, minimum width=2.5cm, minimum height=1cm, text centered, draw=none, fill=#1!30}
  ]

  % Nodes
  \node[rect=white] (gcrf) {\gls{gcrf}};
  \node[rect=white, below right=of gcrf] (eopsq) {$(X,Y)_{CIP} + (\delta X,\delta Y)$};
  \node[rect=red, below=of gcrf] (q) {$\boldsymbol{Q}^T(t) = f(X, Y, s)$};
  \coordinate (mid_gcrf_q) at ($ (gcrf)!0.5!(q) + (-2cm,0) $);
  \node[rectangle, draw=green, fill=none] (cmpq) at (mid_gcrf_q) {$s, (X,Y)_{CIP}$};

  \node[rect=white, below right=of q] (eopsr) {$UT1 = UTC + \Delta UT1$};
  \node[rect=orange, below=of q] (r) {$\boldsymbol{R}^T(t) = R_3(\theta_{ERA})$};
  \coordinate (mid_q_r) at ($ (q)!0.5!(r) + (-2cm,0) $);
  \node[rectangle, draw=green, fill=none] (cmpr) at (mid_q_r) {$\theta _{ERA}$};

  \node[rect=purple, below=of r] (w) {$\boldsymbol{W}^T(t) = R_1(-y_p) R_2(-x_p) R_3(s')$};
  \coordinate (mid_r_w) at ($ (r)!0.5!(w) + (-3cm,0) $);
  \node[rectangle, draw=green, fill=none] (cmpw) at (mid_r_w) {$s', (\delta x,\delta y)_{ocean}, (\delta x,\delta y)_{libration}$};

  \node[rect=white, below=of w] (itrf) {\gls{itrf}};
  \coordinate (mid_w_itrf) at ($ (w)!0.5!(itrf) + (+4cm,0) $);
  \node[rect=white] (eopsw) at (mid_w_itrf) {$(x_p, y_p)_{IERS} + (\delta x,\delta y)_{ocean} + (\delta x,\delta y)_{libration}$};

  % Arrows
  \draw[->, thick] (gcrf) -- (q);
  \draw[->, thick, red] (eopsq) -- (q);
  \draw[->, thick] (q) -- node[right]{\gls{cirs}} (r);
  \draw[->, thick, red] (eopsr) -- (r);
  \draw[->, thick] (r) -- node[right]{\gls{tirs}} (w);
  \draw[->, thick, red] (eopsw) |- (w);
  \draw[->, thick] (w) -- (itrf);

\end{tikzpicture}
\caption{Transformation flowchart from \gls{gcrf} to \gls{itrf}. On the left, inside the green boxes, are the computed quantities needed for each step.}
\label{fig:gcrf_to_itrf}
\end{figure}

\scriptsize{
\begin{tabular}{p{1cm} | p{1.7cm} | p{1.7cm} p{1.7cm} p{1.7cm} | p{1.7cm} p{1.7cm} p{1.7cm}}
 Transformation 
  & \gls{tirs}-to-\gls{itrs} 
  & \multicolumn{3}{c}{\gls{gcrs}-to-\gls{cirs}} 
  & \multicolumn{3}{c}{\gls{gcrs}-to-\gls{itrs}} \\
 \hline
 SOFA 
  & iauPom00 
  & \multicolumn{3}{c}{iauPn} 
  & \multicolumn{3}{c}{iauC2t06a} \\
 Lib\footnote{Functions are in the \texttt{dso::detail} namespace.} 
  & \index{tirs2i}{\texttt{tirs2i}} 
  & \index{C}{\texttt{C}} 
  & \index{C\_impl}{\texttt{C\_qimpl}} 
  & \index{C\_rxyimpl}{\texttt{C\_rxyimpl}} 
  & \index{c2i06a}{\texttt{c2i06a}} 
  & \index{c2i06a\_bz}{\texttt{c2i06a\_bz}}
  & \index{c2tirs}{\texttt{c2tirs}} \& \index{tirs2i}{\texttt{tirs2i}} \\
 SOFA-Lib [m] & $10^{-7}$ \gls{meo}\newline $10^{-8}$ \gls{leo} 
              & $10^{-7}$ \gls{meo}\newline $10^{-8}$ \gls{leo} 
              & $10^{-7}$ \gls{meo}\newline $10^{-8}$ \gls{leo}
              & $10^{-8}$ \gls{meo}\newline $10^{-8}$ \gls{leo} 
              & $10^{-3}$ \gls{meo}\newline $10^{-4}$ \gls{leo} 
              & $10^{-3}$ \gls{meo}\newline $10^{-4}$ \gls{leo}
              & $10^{-3}$ \gls{meo}\newline $10^{-4}$ \gls{leo} \\
 \hline
\end{tabular}
}
