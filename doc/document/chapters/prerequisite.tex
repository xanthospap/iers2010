\section{Installation and Prerequisites}\label{sec:installation-and-prerequisites}

Source code is ISO C++17. Compilation should be trivial using any C++ compiler
supporting the c++17 standard (option \texttt{-std=c++17} in \href{https://gcc.gnu.org/}{gcc}
and \href{https://clang.org/}{clang}).

\subsection{Third Party Libraries}\label{ssec:third-party}
\begin{description}
  \item [\href{https://eigen.tuxfamily.org/index.php?title=Main_Page}{Eigen}]; 
  used for Vector/Matrix operations.
  \item [\href{https://naif.jpl.nasa.gov/naif/toolkit.html}{SPICE Toolkit}]; 
  used for extracting planet positions off from. 
  \href{https://ssd.jpl.nasa.gov/planets/eph_export.html}{JPL Planetary Ephemerides} files 
  (see \autoref{sec:moonandplanetaryephemeris}).
\end{description}

Installation of \href{https://eigen.tuxfamily.org/index.php?title=Main_Page}{Eigen} is 
pretty trivial (it is actually a header-only file). Download the source code and follow 
the instructions in \href{https://gitlab.com/libeigen/eigen/-/blob/master/INSTALL?ref_type=heads}{INSTALL}.

\href{https://naif.jpl.nasa.gov/naif/toolkit.html}{SPICE Toolkit} is a C-library. 
To install it (system-wide)
\begin{enumerate}
  \item Download the C library from the \href{https://naif.jpl.nasa.gov/naif/toolkit_C.html}{official repository} 
    and uncompress.
  \item Use the script \mpath{script/cppspice/c2cpp_header.py} to tranform 
    C header files; run the script using the cspice \mpath{include} folder path 
    as command line argument. I.e. if the uncompressed cspice folder is at \mpath{/home/work/var/cspice}, 
    use \texttt{c2cpp\_header.py /home/work/var/cspice}.
  \item Run the \mpath{makeall.csh} script provided by the distribution (under 
    \mpath{cspice} folder). Note that the script is in the cshell, hence you 
    might need to \texttt{csh makeall.csh}.
  \item Copy the \mpath{script/cppspice/install.sh} script under the \mpath{cspice} 
    folder; run it as root, to install the library. Header files will be available at 
    \mpath{/usr/local/include} and the library at \mpath{/usr/local/lib}.
\end{enumerate}

\subsection{Building the Project}
Building the project is done via \href{https://cmake.org/}{cmake}. If you just need a clean build, and have already 
installed dependencies (\ref{ssec:third-party}), go to the \texttt{ROOT} directory and just type:
\begin{lstlisting}[style=shell, label={lst:basic-build}]
  cmake -S . -B build -DCMAKE_BUILD_TYPE=Release -DCMAKE_PREFIX_PATH=/usr/local/lib
  cmake --build build --target all --config=Release -- -j4
  cd build && sudo make install
\end{lstlisting}

\noindent\textbf{Debug Mode}: To build the project in \textbf{Debug} mode, change \texttt{Release} to \texttt{Debug} 
in \ref{lst:basic-build}.

\subsubsection{Building the Tests}
There are a series of tests that can be build, organized in different folders. Each of these are build depending on 
different options and Prerequisites. In general, to build any of the test stacks, you will have to turn on the 
\texttt{BUILD\_TESTING} option (i.e. \texttt{cmake -S . [...] -DBUILD\_TESTING=ON}). If the build is successeful, you 
can trigger the tests via the command \texttt{ctest --test-dir build}.

\noindent\textbf{COSTG Benchmark Tests}: These tests are based on \cite{Lasser2020} and are a list of programs 
that check the gravitational forces acting on a daily arc of the GRACE satellite. To compile these tests, you will need 
to download the test data from the location specified in \cite{Lasser2020}. Once you have them in place, you can trigger 
the building of the test programs, via the \texttt{BUILD\_COSTG} option and (optionally) set the \texttt{COSTG\_DATA\_DIR} 
variable to the location where the (downloaded) test data are to be found. Example:
\begin{lstlisting}[style=shell, label={lst:basic-build}]
  cmake -S . -B build -DCMAKE_BUILD_TYPE=Release -DCMAKE_PREFIX_PATH=/usr/local/lib \
    -DCOSTG_DATA_DIR=/foo/bar -DBUILD_COSTG=ON -DBUILD_TESTING=ON
  cmake --build build --target all --config=Release -- -j4 
  # run the tests
  ctest --test-dir build
\end{lstlisting}

\noindent\textbf{Unit Tests}: Unit tests are build by default, if the \texttt{BUILD\_TESTING} option is set 
to \texttt{ON}.

\noindent\textbf{SOFA Tests}: These tests are meant to validate various algorithms against the SOFA \cite{sofa2021}
library (C implementation). They are build by default if :
\begin{itemize}
  \item the \texttt{BUILD\_TESTING} option is set to \texttt{ON}, and
  \item the SOFA library is available on your system (as \texttt{sofa\_c})
\end{itemize}

\subsection{Build Options for Developers}
\begin{description}
  \item[SPARSE\_ADMITTANCE\_MATRIX] uses sparse matrices (instead of dense ones) on a few different locations 
  (e.g. ocean tides modeling)
\end{description}